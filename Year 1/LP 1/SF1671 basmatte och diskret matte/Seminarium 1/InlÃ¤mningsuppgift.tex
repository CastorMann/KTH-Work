\documentclass{article}
\begin{document}
  1. Hitta alla lösningar till ekvationen
  \begin{equation}
      cos^2(4x + \frac{\pi}{3}) = sin^2(3x-\frac{\pi}{4})
  \end{equation}
  
  2. Hitta alla lösningar till ekvationen
  \begin{equation}
      |2x^3 + x^2 - 13x + 6| = 1 - 2x
  \end{equation}
  Vi delar in ekvationen i dess två möjliga fall:
  \begin{equation}
      2x^3 + x^2 - 13x + 6 = 1 - 2x
  \end{equation}
  och 
  \begin{equation}
      2x^3 + x^2 - 13x + 6 = 2x - 1
  \end{equation}
  Vi börjar med den första möjligheten, ekvation (3). Subtrahera 1 - 2x från bägge sidor.
  \newline \newline
  $2x^3 + x^2 - 11x + 5 = 0$
  \newline \newline
  Vi faktoriserar vänsterledet och får
  \newline \newline
  $(2x - 1)(x^2 + x - 5) = 0$
  \newline \newline
  Ekvationen har 3 rötter. En finns då $2x - 1$ är 0 och övriga då $x^2 + x - 5$ är 0. Den förstnämnda roten är trivial och övriga finns exempelvis via pq-formeln. Rötterna är följande: $x = \frac{1}{2}$, $x = \frac{\sqrt{21}}{2} - \frac{1}{2}$ och $x = -\frac{\sqrt{21}}{2} - \frac{1}{2}$
  \newline \newline
  Vi repeterar stegen ovan för den andra ekvationen, ekvation (4). Subtrahera 2x - 1 från bägge sidor och faktorisera vänsterledet. Vi får följande ekvation:
  \newline \newline
  $(2x - 1)(x^2 + x - 7) = 0$
  \newline \newline
  Vi har samma triviala rot $x = \frac{1}{2}$ även i denna ekvation. Övriga rötter finner vi på samma sätt som tidigare. Dessa rötter är följande: $x = \frac{\sqrt{29}}{2} - \frac{1}{2}$ och $x = -\frac{\sqrt{29}}{2} - \frac{1}{2}$
  \newline \newline
  Vi har alltså funnit 5 rötter: $x = \frac{1}{2}$, $x = \frac{\sqrt{21}}{2} - \frac{1}{2}$, $x = -\frac{\sqrt{21}}{2} - \frac{1}{2}$, $x = \frac{\sqrt{29}}{2} - \frac{1}{2}$ och $x = -\frac{\sqrt{29}}{2} - \frac{1}{2}$. Vi måste nu verifiera dessa genom att sätta in värdena för x i vår ursprungliga ekvation. När vi gör detta finner vi att två av rötterna är falska: $x = \frac{\sqrt{21}}{2} - \frac{1}{2}$ och $x = \frac{\sqrt{29}}{2} - \frac{1}{2}$. Våra lösnignar är således följande: $x = \frac{1}{2}$, $x = -\frac{\sqrt{21}}{2} - \frac{1}{2}$ och $x = -\frac{\sqrt{29}}{2} - \frac{1}{2}$
  \newline \newline
  Svar: $x = \frac{1}{2}$ eller $x = -\frac{\sqrt{21}}{2} - \frac{1}{2}$ eller $x = -\frac{\sqrt{29}}{2} - \frac{1}{2}$
  
\end{document}