\documentclass{article}
\begin{document}
  1. Hitta alla lösningar till ekvationen
  \begin{equation}
      cos^2(4x + \frac{\pi}{3}) = sin^2(3x-\frac{\pi}{4})
  \end{equation}
  Vi börjar med notera att $sin(v) = cos(\frac{\pi}{2} - v)$ och skriver således om ekvationen till: \newline\newline
  $cos^2(4x + \frac{\pi}{3}) = cos^2(\frac{3\pi}{4} - 3x)$\newline\newline
  Lösningarna till denna ekvation finns då $4x + \frac{\pi}{3} = \pi n \pm (\frac{3\pi}{4} - 3x)$ för alla $n \in \mathBB{Z}$. Vi kan dela in ekvationen i dess två möjligheter:\newline\newline
  $4x + \frac{\pi}{3} = \pi n + \frac{3\pi}{4} - 3x$ och $4x + \frac{\pi}{3} = \pi n - \frac{3\pi}{4} + 3x$.\newline\newline
  Den första av dessa har lösnignarna $x = \frac{\pi n}{7} + \frac{5\pi}{84}$ och den andra har lösningarna $x = \pi n - \frac{13\pi}{12}$\newline\newline
  Svar: $x = \frac{\pi n}{7} + \frac{5\pi}{84}$ eller $x = \pi n - \frac{13\pi}{12}$ för alla $n \in \mathBB{Z}$.
  \newline \newline2. Hitta alla lösningar till ekvationen. 
  \begin{equation}
      |2x^3 + x^2 - 13x + 6| = 1 - 2x
  \end{equation}
  Vi delar in ekvationen i dess två möjliga fall:
  \begin{equation}
      2x^3 + x^2 - 13x + 6 = 1 - 2x
  \end{equation}
  och 
  \begin{equation}
      2x^3 + x^2 - 13x + 6 = 2x - 1
  \end{equation}
  Vi börjar med den första möjligheten, ekvation (3). Subtrahera 1 - 2x från bägge sidor. \newline \newline
  $2x^3 + x^2 - 11x + 5 = 0$ \newline \newline
  Vi faktoriserar vänsterledet och får \newline \newline
  $(2x - 1)(x^2 + x - 5) = 0$ \newline \newline
  Ekvationen har 3 rötter. En finns då $2x - 1$ är 0 och övriga då $x^2 + x - 5$ är 0. Den förstnämnda roten är trivial och övriga finns exempelvis via pq-formeln. Rötterna är följande: $x = \frac{1}{2}$, $x = \frac{\sqrt{21}}{2} - \frac{1}{2}$ och $x = -\frac{\sqrt{21}}{2} - \frac{1}{2}$ \newline \newline
  Vi repeterar stegen ovan för den andra ekvationen, ekvation (4). Subtrahera 2x - 1 från bägge sidor och faktorisera vänsterledet. Vi får följande ekvation:\newline \newline
  $(2x - 1)(x^2 + x - 7) = 0$ \newline \newline
  Vi har samma triviala rot $x = \frac{1}{2}$ även i denna ekvation. Övriga rötter finner vi på samma sätt som tidigare. Dessa rötter är följande: $x = \frac{\sqrt{29}}{2} - \frac{1}{2}$ och $x = -\frac{\sqrt{29}}{2} - \frac{1}{2}$ \newline \newline
  Vi har alltså funnit 5 rötter: $x = \frac{1}{2}$, $x = \frac{\sqrt{21}}{2} - \frac{1}{2}$, $x = -\frac{\sqrt{21}}{2} - \frac{1}{2}$, $x = \frac{\sqrt{29}}{2} - \frac{1}{2}$ och $x = -\frac{\sqrt{29}}{2} - \frac{1}{2}$. Vi måste nu verifiera dessa genom att sätta in värdena för x i vår ursprungliga ekvation. När vi gör detta finner vi att två av rötterna är falska: $x = \frac{\sqrt{21}}{2} - \frac{1}{2}$ och $x = \frac{\sqrt{29}}{2} - \frac{1}{2}$. Våra lösnignar är således följande: $x = \frac{1}{2}$, $x = -\frac{\sqrt{21}}{2} - \frac{1}{2}$ och $x = -\frac{\sqrt{29}}{2} - \frac{1}{2}$ \newline \newline
  Svar: $x = \frac{1}{2}$ eller $x = -\frac{\sqrt{21}}{2} - \frac{1}{2}$ eller $x = -\frac{\sqrt{29}}{2} - \frac{1}{2}$\newline \newline
  
  3. Hitta alla $x \in \mathbb{R}$ sådana att 
  \begin{equation}
      ln(4 + e^{2x}) \leq ln(1 + 2e^x) + ln(3 + e^x)
  \end{equation}
  Vi börjar med att upphöja e till bägge leden för att få bort alla logaritmer:\newline
  $4 + e^{2x} \leq (1 + 2e^x)(3 + e^x) \Rightarrow 4 + e^{2x} \leq 3 + 7e^x + 2e^{2x} \Rightarrow (e^x)^2 + 7e^x - 1 \geq 0$\newline pq-formeln ger sedan att $e^x \geq -\frac{7}{2} + \sqrt{(\frac{7}{2})^2 + 1} = \frac{\sqrt{65} - 7}{2}$. Notera att vi inte är intresserade av den negativa lösningen som pq-formeln vanligvis ger eftersom $e^x$ aldrig kan vara negativt. Slutligen kan vi beräkna x genom att ta den naturliga logaritmen på bägge sidor.\newline \newline
  Svar: $\{x \in \mathbb{R}: x \geq ln(\frac{\sqrt{65} - 7}{2})\}$\newline
  
  4. Lös ekvationen 
  \begin{equation}
      arctan(sinh(x)) = -\frac{\pi}{4}
  \end{equation}
  Vi tar tangens för bägge sidor. och får $sinh(x) = tan(-\frac{\pi}{4})$. $tan(-\frac{\pi}{4})$ har den triviala lösningen -1. Vi tar sedan arcsinh på bägge sidor och får $x = arcsinh(-1) \Rightarrow x = -arcsinh(1)$ \newline \newline
  Svar: $x = -arcsinh(1)$
  
\end{document}