\documentclass{article}
\usepackage[utf8]{inputenc}

\title{Mängdlära}

\begin{document}

\maketitle

\section{Föreläsning 1}
En mängd är en samling element. Mängder kan beskrivas genom att:
\newline* lista elementen: \{1, -7, $\sqrt{3}$\}
\newline* beskriva mängden: \{$\alpha \in \mathbb{R}: \alpha \geq f(\alpha)$\}
\newline\newline
\begin{tabular}{|c|c|}
\hline
    $\mathbb{N}$ & set of natural numbers \\\hline
    $\mathbb{Z}$ & set of integer numbers \\\hline
    $\mathbb{Q}$ & set of rational numbers \\\hline
    $\mathbb{R}$ & set of real numbers \\\hline
    $\mathbb{C}$ & set of complex numbers \\\hline
    $\in$ & is member of \\\hline
    $\notin$ & is not member of \\\hline
    $\ni$ & owns (has member) \\\hline
    $\subset$ & is proper subset of \\\hline
    $\subseteq$ & is subset of \\\hline
    $\cup$ & set union \\\hline
    $\cap$ & set intersection \\\hline
    $\mathbb{P}(A)$ & powerset of A (all subsets of A) \\\hline
\end{tabular}
\newline\newline
Storleken av potensmängden till en mängd, A, är 2 upphöjt till storleken av mängden A: $|P(A)| = 2^{|A|}$

\end{document}
